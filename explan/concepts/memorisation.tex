For many problems, some calculations will be repeated many times. This is a waste of resources, so we can stored the calculations for laster. Consider the example below for calculating the Fibonacii numbers.

\begin{lstlisting}
fib :: Integral a => a -> a
fib 1 = 1
fib 0 = 0
fib n = (fib $ n-1) + (fib $ n-2)
\end{lstlisting}

All calls to \textit{fib} with a number greater than $n$ calls \textit{fib n}. This can become very wasteful if we had to calculate \textit{fib} often. Hence we can stored intermediate results for later collection. This is done as follows.

\begin{lstlisting}
import Data.Array

fib :: Integral a => a -> a
fib n = dp ! n
	where dp = array (0,limit) [(i,f i) | i <- [0 .. limit]]
	      f 0 = 0
	      f 1 = 1
	      f n = (fib $ n - 1) + (fib $ n -2)
\end{lstlisting}

The important work is done here by Haskell's inbuilt laziness and the storage array. When \textit{fib n} is called, it looks up the $n^\text{th}$ value in array \textit{dp}. This gives the value of our function \textit{f}, which is defined the same as our old \textit{fib} function. Because of laziness, the array starts of as an array of thunks: it doesn't calculate the function until called. So, when the function is called the first time it is calculated, and from then on the answer is simply retrieved.

This method has some drawbacks. Firstly, the entire calculation must be stored in memeory. Secondly, the range of the calculation must be calculated in advnce. It is safe to use a high upper limit, as calls aren't used until called, but this is an extra constraint.
